\documentclass{article}
\usepackage{biblatex} %Imports biblatex package
\addbibresource{refs.bib} %Import the bibliography file
\usepackage[utf8]{inputenc}

\usepackage[top=1in, bottom=1in, left=1in, right=1in]{geometry}

\usepackage[onehalfspacing]{setspace}

\usepackage{amsmath, amssymb, amsthm}
\usepackage{enumerate, enumitem}
\usepackage{fancyhdr, graphicx, proof, comment, multicol}
\usepackage[none]{hyphenat}
\usepackage{dirtytalk}
\binoppenalty=\maxdimen
\relpenalty=\maxdimen
\usepackage{microtype}
%\usepackage{mathpazo}
\usepackage{mdframed}
\usepackage{parskip}
\linespread{1.1}
\usepackage{graphicx}
\usepackage{subfig}
\usepackage{mathrsfs}
\usepackage{amsfonts}
\usepackage{amsmath}
\usepackage{minted}
\usepackage{amssymb}
\usepackage{algorithm}
\usepackage[noend]{algpseudocode}
\usepackage{mathtools}
\usepackage{hyperref}
\usepackage{multicol}
\usepackage{tcolorbox}
\usepackage{array}
\usepackage{colortbl}
\usepackage{booktabs}
\usepackage{diagbox}

\tcbuselibrary{minted,breakable,xparse,skins}

\definecolor{bg}{gray}{0.95}
\DeclareTCBListing{mintedbox}{O{}m!O{}}{%
  breakable=true,
  listing engine=minted,
  listing only,
  minted language=#2,
  minted style=default,
  minted options={%
    linenos,
    gobble=0,
    breaklines=true,
    breakafter=,,
    fontsize=\small,
    numbersep=8pt,
    #1},
  boxsep=0pt,
  left skip=0pt,
  right skip=0pt,
  left=25pt,
  right=0pt,
  top=3pt,
  bottom=3pt,
  arc=5pt,
  leftrule=0pt,
  rightrule=0pt,
  bottomrule=2pt,
  toprule=2pt,
  colback=bg,
  colframe=orange!70,
  enhanced,
  overlay={%
    \begin{tcbclipinterior}
    \fill[orange!20!white] (frame.south west) rectangle ([xshift=20pt]frame.north west);
    \end{tcbclipinterior}},
  #3}



\title{Intel·ligència Artificial Distribuïda: Informe Practica 2}
\author{Eshaan Mittal, Adrià Gasull}
\date{Diciembre 2025}
\begin{document}

\maketitle

\begin{center}
    \includegraphics[width=0.7\textwidth]{images/title.png}
\end{center}
\newpage

\tableofcontents

\newpage

\section{Introduction}

This report presents the implementation of a multi-agent system for a Dutch Fish Auction simulation. The project explores three different approaches to agent design and coordination: a standard implementation using the \texttt{osBrain} framework, an LLM-augmented implementation where agents use Large Language Models for decision-making, and a LangGraph-based implementation for more complex agent workflows.

The core scenario involves a distributed marketplace with multiple Sellers (Operators) and Buyers (Merchants). Operators conduct Dutch auctions, where the price of a fish starts high and decreases over time. Merchants, each with specific preferences and budgets, must decide when to bid to maximize their utility while competing with others. This creates a dynamic environment requiring real-time decision-making, race condition handling, and strategic planning.

\section{osBrain Implementation}

The first implementation utilizes the \texttt{osBrain} Python framework, which provides a flexible infrastructure for creating and connecting agents via message passing (using ZeroMQ/PyRO). This section details the architecture, design decisions, and behavior of the agents implemented in \texttt{toyAgentOsBrain.py}.

\subsection[Design Decisions]{Design Decisions}

\subsubsection{Agent Architecture}
Two types of agents were defined:
\begin{itemize}
    \item \textbf{Operator (Seller):} Manages an inventory of fish (Hake, Sole, Tuna) with randomized starting and minimum prices. It runs an independent auction clock, broadcasting price updates and processing buy requests.
    \item \textbf{Merchant (Buyer):} Connects to all operators simultaneously. It maintains a budget and a set of goals (Diversity, Preference Satisfaction) and makes bidding decisions based on a specific strategy.
\end{itemize}

\subsubsection{Communication Patterns}
We employed two primary communication patterns to ensure efficient and scalable interaction:
\begin{itemize}
    \item \textbf{PUB-SUB (Publish-Subscribe):} Operators use a \texttt{PUB} socket to broadcast \texttt{AUCTION\_ITEM} updates and \texttt{SALE\_CONFIRMATION} messages. Merchants use \texttt{SUB} sockets to listen to these broadcasts. This allows one-to-many communication without the operator needing to know about specific merchants.
    \item \textbf{PUSH-PULL (Pipeline):} Merchants use \texttt{PUSH} sockets to send targeted \texttt{BUY} requests to specific Operators, who receive them via a \texttt{PULL} socket. This pattern queues requests, enabling the Operator to process them sequentially and handle race conditions (first valid bid wins).
\end{itemize}

\subsubsection{Bidding Logic}
The merchants employ a prioritized goal-oriented bidding logic:
\begin{enumerate}
    \item \textbf{Priority 1: Diversity (Get Missing Types):} If the merchant does not yet own a fish of the current type, it will bid aggressively. It reserves budget for other missing types and is willing to spend up to 40\% of its remaining available budget.
    \item \textbf{Priority 2: Preference Satisfaction:} If the merchant already has the fish type but it matches its specific preference, it will bid if the price is reasonable (up to 60\% of the available budget).
    \item \textbf{Priority 3: Opportunistic (Bargains):} If the price drops very low ($\le$ 15), the merchant will bid on any fish type to maximize value, regardless of its goals.
\end{enumerate}

\subsubsection{Race Condition Handling}
Since multiple merchants can bid on the same item simultaneously, we implemented a robust mechanism to handle race conditions:
\begin{enumerate}
    \item Operators process incoming bids sequentially.
    \item The first valid bid triggers an atomic \texttt{is\_sold} flag.
    \item Subsequent bids for the same item are ignored.
    \item A \texttt{SALE\_CONFIRMATION} is broadcast to ALL merchants.
    \item Merchants receiving the confirmation update their state: the winner deducts the budget, while losers clear their pending bids and prepare for the next item.
\end{enumerate}

\subsection[Tests]{Tests}

The simulation was tested with 2 Operators and 3 Merchants. Logs were generated to verify the system's behavior:
\begin{itemize}
    \item \texttt{setup\_*.csv}: confirmed that merchants were correctly assigned random preferences.
    \item \texttt{log\_*.csv}: recorded the flow of the auction.
\end{itemize}

The message flow correctly handled concurrent bids, ensuring no single item was sold twice. The diversity of merchant preferences resulted in a realistic bidding pattern where different merchants competed for different items.

\subsection[Summary]{Summary}

The \texttt{osBrain} implementation successfully demonstrates a distributed Dutch auction. The use of asynchronous message passing creates a robust and dynamic marketplace. The separation of concerns between Operators and Merchants, along with the specific communication patterns chosen, ensures scalability and correct handling of concurrency issues.

\section{LLM-Augmented Implementation}


\subsection[Design Decisions]


\subsection[Tests]


\subsection[Summary]


\section{LangGraph Implementation}


\subsection[Design Decisions]


\subsection[Tests]


\subsection[Summary]


\section{Conclusions}


\end{document}